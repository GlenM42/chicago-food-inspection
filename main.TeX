\documentclass[12pt]{article}
\usepackage{geometry}
\usepackage{amsmath}
\usepackage{graphicx}
\usepackage{hyperref}

\geometry{a4paper, margin=1in}

\title{Proposal Document: Chicago Food Inspection Data Analysis}
\author{Glen Miasnychenko}
\date{}

\begin{document}

\maketitle

\section*{Introduction and Overview of the Project Scope}

\hspace*{2em}The primary objective of this project is to analyze the Chicago food inspection dataset to derive actionable insights.
The analysis will explore inspection trends, identify compliance patterns, and detect potential biases across different
dimensions, including geographic regions, time periods, and facility types.

\section*{Data Loading and Pre-Processing Strategies}

\subsection*{Data Loading}
The dataset (\texttt{Food\_Inspections.csv}) will be loaded using the \texttt{pandas.read\_csv()} method.
Key columns and their expected datatypes include:
\begin{itemize}
    \item \textbf{DBA Name (string):} Name of the food establishment.
    \item \textbf{Address (string):} Location details.
    \item \textbf{Zip (float64):} Zip code of the establishment (to be converted to \texttt{int32}).
    \item \textbf{Inspection Date (datetime64[ns]):} Timestamp of inspections.
    \item \textbf{Results (category):} Inspection outcomes.
    \item \textbf{Violations (string):} Description of violations.
    \item \textbf{Latitude and Longitude (float64):} Geographic coordinates for spatial analysis.
    \item \textbf{License \# (int64):} Unique identifier for establishments.
\end{itemize}

\subsection*{Pre-Processing}
After loading the document, we expect some level of pre-processing to be required:
\begin{itemize}
    \item Convert \texttt{Inspection Date} to \texttt{datetime} format for improved time-series analysis.
    \item Standardize \texttt{Zip} as \texttt{int32} for geospatial alignment.
    \item Handle missing data:
    \begin{itemize}
        \item Drop rows with critical missing fields (\texttt{License \#}, \texttt{Zip}, \texttt{Inspection Date}).
        \item Impute empty \texttt{Violations} with ``No violations reported.''
    \end{itemize}
    \item Clean textual columns (\texttt{DBA Name}, \texttt{Violations}) by trimming whitespace and converting to lowercase.
\end{itemize}

\section*{Topics to Cover}
\subsection*{High and Low Pass Rate Zip Codes}
\textbf{Objective:} Identify zipcodes with the highest and lowest pass rates to understand regional compliance trends.

\textbf{Approach:}
\begin{itemize}
    \item Copy the related dataframe section (columns \textit{Zip} and \textit{Results}).
    \item Create a \textit{pass categories} and \textit{fail categories} that would consist of related values.
    \item Create a derived column, \texttt{Pass\_Fail} (\texttt{int8}):
    \begin{itemize}
        \item 1 for \texttt{Pass} or \texttt{Pass w/ Conditions}.
        \item 0 for \texttt{Fail} or \texttt{Out of Business}.
    \end{itemize}
    \item Drop the na values.
    \item Group data by \texttt{Zip} and calculate the mean of \texttt{Pass\_Fail} to derive pass rates, and convert to percentage.
    \item Sort the new dataframe by the pass rate in descending order.
    \item Display both the table with the top 5 zipcodes by the pass rates, and 5 zipcodes with the lowest rates.
    \item The next step would be to display a graph with the pass rates as \textbf{y-axis} and the zipcodes as the \textbf{x-axis}.
    \item As a lot of people have no knowledge of the Chicago zipcodes.
    It would be great to show the heatmap with the actual map of the zipcodes (that would require a geojson file).
\end{itemize}

\textbf{Visualization:}
\begin{itemize}
    \item Table with the highest top 5 and lowest top 5 values.
    \item Bar chart showing pass rates by zipcode.
    \item Heat map of pass rates using \texttt{geopandas} and a GeoJSON file for Chicago zipcode boundaries.
\end{itemize}

\subsection*{Pass Rate Changes Over Time}
\textbf{Objective:} Analyze trends in pass rates to understand temporal improvements or challenges.

\textbf{Approach:}
\begin{itemize}
    \item Convert the \texttt{Inspection Date} column into the datetime format, if not done before.
    \item Extract \texttt{Year-Month} (\texttt{period[M]}) from \texttt{Inspection Date}.
    \item Group inspections by \texttt{Year-Month} and calculate average \texttt{Pass\_Fail} rates.
    \item Plot it using \texttt{matplotlib}.
\end{itemize}

\textbf{Visualization:}
\begin{itemize}
    \item Line plot using \texttt{matplotlib} to display pass rate changes over time.
\end{itemize}

\subsection*{Trend in Number of Inspections}
\textbf{Objective:} Explore inspection frequency trends across time and zipcodes.

\textbf{Approach:}
\begin{itemize}
    \item Count inspections by \texttt{Year-Month} and by \texttt{Zip}.
    \item Highlight trends for the 10 zipcodes with the highest inspection counts.
    \item Print both graphs.
\end{itemize}

\textbf{Visualization:}
\begin{itemize}
    \item A line chart showing inspection trends over time.
    \item Multi-line chart showing the inspection trends for 10 zipcodes.
\end{itemize}

\subsection*{Common Reasons for Violations}
\textbf{Objective:} Identify the most frequent violations to target recurring compliance issues.

\textbf{Approach:}
\begin{itemize}
    \item Get the rows with non-empty \texttt{Violations} column.
    \item Tokenize the \texttt{Violations} column into individual reasons, splitting by \texttt{|}.
    \item Count occurrences using \texttt{collections.Counter}.
    \item Get the ten the most common violations using the \texttt{most\underline{\space}common} method.
    \item Extract violation codes (numbers before the first \texttt{.}) for easier reference.
    \item Print the table with the full description of the violations.
    \item Display the chart with the top 10 most common codes of violations.
\end{itemize}

\textbf{Visualization:}
\begin{itemize}
    \item Horizontal bar chart showing the top 10 most common codes of violations.
    \item Table linking violation codes to full descriptions.
\end{itemize}

\subsection*{Repeated Violations at Locations}
\textbf{Objective:} Highlight establishments with repeated violations across multiple inspections.

\textbf{Approach:}
\begin{itemize}
    \item Group data by \texttt{License \#}, excluding the establishments with invalid Licenses.
    \item Filter establishments with repeated violations (more than one).
    \item Add the regular information about a business to a completed dataframe.
    \item Display the table with top 10 establishments with repeated violations.
    \item Graph the number of violations against the name of the business.
\end{itemize}

\textbf{Visualization:}
\begin{itemize}
    \item Bar chart of the top 10 establishments with the most repeated violations.
    \item Table with more detailed information about said businesses.
\end{itemize}

\subsection*{Inspections for a Given Restaurant Name}
\textbf{Objective:} Provide a detailed inspection history for a specified restaurant.

\textbf{Approach:}
For the ease of use, we will try to create a function, which will take the \texttt{restaurant\underline{\space}name} as an argument,
and output the history of violations, and a graph.
The function is supposed to work in this fashion:
\begin{itemize}
    \item Filter data for the \texttt{DBA Name} using case-insensitive string matching.
    \begin{description}
        \item[Note:] If there are no matches found, we will return nothing, and output a string \textit{"No inspections found"}
    \end{description}
    \item Group the data by month in \texttt{Inspection Date} because the info by days will become too big.
    \item Print the table with the relevant columns, such as \texttt{Results}, \texttt{Violations}, and \texttt{Risk}.
    \item Graph the \texttt{Results} column against the month in a bar chart, displaying \textit{Pass} as a green bar, and \textit{Fail} as red.
    Any other results may be yellow.
\end{itemize}

\textbf{Visualization:}
\begin{itemize}
    \item Table with the main results.
    \item Stacked bar chart showing inspection outcomes over time.
\end{itemize}

\subsection*{Nearby Restaurants for a Given Address}
\textbf{Objective:} Identify restaurants near a specific address and compare their inspection outcomes.

\textbf{Approach:}
Just as in the previous section, I believe it would be the most fitting to do the section as a function that takes address and radius as the arguments.
The main qualities of said function should be:
\begin{itemize}
    \item Ensure the \texttt{Latitude} and \texttt{Longitude} are present, and display an info message in case they are not.
    \item Use \texttt{geopy} to geocode the given address into \texttt{Latitude} and \texttt{Longitude}.
    \item Calculate distances to all establishments using \texttt{geopy.distance.geodesic}.
    \item Filter establishments within a specified radius (e.g., 1 km).
    \begin{description}
        \item[Note:] It is important to make sure we have error messages in case there are no establishments in a given radius.
    \end{description}
    \item Once the search is completed, we need to display a list of the closest establishments from closest to the farthest.
\end{itemize}

\textbf{Visualization:}
\begin{itemize}
    \item Table showing nearby restaurants and their inspection results.
\end{itemize}

\subsection*{Potential Bias in Inspection Frequency}
\textbf{Objective:} Check if there are any zipcodes receiving significantly more attention from the inspection agency.

\textbf{Approach:}
I believe the best metric we can use for this section is going to be the normalized number of inspections per zipcode.
The process would look like this:
\begin{itemize}
    \item Calculate inspection counts (\texttt{int64}) per zipcode.
    \begin{description}
        \item[Note:] While the main focus of this section is \textit{normalized numbers of inspections per zipcode}, I believe the sheer number of inspections by Zip Codes is also interesting.
        Therefore, I believe it is useful to display that in a way of a barchart.
    \end{description}
    \item Calculate the number of business establishments per zipcode.
    \item Normalize counts of inspections (using unique \texttt{License \#} counts).
    \item Display it in the form of a barchart using \texttt{matplotlib}.
    \item Again, the pure numbers of zipcodes are not necessarily informative.
    The more useful visualization is a heatmap that uses \texttt{geojson}.
    I plan to use \href{https://github.com/smartchicago/chicago-atlas}{this geojson} file.
\end{itemize}

\textbf{Visualization:}
\begin{itemize}
    \item Bar chart of normalized inspection frequencies.
    \item Heat map using \texttt{geopandas} to display normalized rates geographically.
\end{itemize}

\section*{Conclusion}
\hspace*{2em}This proposal provides a robust framework for analyzing Chicago's food inspection data.
By employing systematic data handling, geospatial analysis, and detailed visualizations, this project will address public health concerns and highlight areas for regulatory improvement.

\end{document}